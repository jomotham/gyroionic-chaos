% Math Packages
\usepackage{amssymb, amsmath, amsfonts, amsthm, mathtools, graphicx, tcolorbox, cancel, mathrsfs}

% Formatting Packages
\usepackage{cite, pdfpages, tikz, pgffor, float, pgfplots, import, xcolor, transparent, siunitx, caption, enumitem, titlesec, multicol, pifont, verbatim, fancyvrb, calc, xspace}


\usepackage{wrapfig,paralist,rotating,subfig}
\usepackage[textwidth=6in,textheight=8in]{geometry}

% Please use vector graphics in .pdf format where possible
% This is much preferable to large bitmap files
\DeclareGraphicsExtensions{{.pdf},{.jpg},{.png}}

\usepackage[english]{babel}
% \usepackage[utf8]{inputenc}
\usepackage[dvipsnames]{xcolor}
\usepackage[absolute]{textpos}

\usepackage{blindtext}


\usetikzlibrary{shapes.geometric}
\pgfplotsset{compat=1.18}
\pdfsuppresswarningpagegroup=1
\tikzset{every picture/.style={line width=0.75pt}}

\renewcommand{\baselinestretch}{1.5}

\newcommand{\eref}[2][]{%
	\ifthenelse{\equal{#1}{}}%
	{Eq.~(\ref{eq:#2})}%
	{Equation~(\ref{eq:#2})}\xspace}

% To refer to a figure, use \fref{<label>}, where <label> is the
% part that follows the 'fig:' in the label of the figure.
\newcommand{\fref}[2][]{%
  \ifthenelse{\equal{#1}{}}%
	{Fig.~\ref{fig:#2}}%
	{Figure~\ref{fig:#2}}}

% How to type vectors. If you don't like this style, you can just change
% the definition of \VEC!!
\newcommand{\VEC}[1]{\ensuremath{\boldsymbol{#1}}\xspace}

% unit vectors
\newcommand{\ux}{\ensuremath{\mathbf{\hat{x}}}\xspace}
\newcommand{\uy}{\ensuremath{\mathbf{\hat{x}}}\xspace}
\newcommand{\uz}{\ensuremath{\mathbf{\hat{x}}}\xspace}

% theta dot and double dot
\newcommand{\thd}{\ensuremath{\dot{\theta}}\xspace}
\newcommand{\thdd}{\ensuremath{\ddot{\theta}}\xspace}

% Derivatives. Some publications require the d in a differential
% to be set in roman. I've never liked that, but I have trained myself
% always to use these two macros, so if I need to switch, it is
% painless.
\newcommand{\DD}{\ensuremath{d}}		% differential d without leading space
\newcommand{\dd}{\ensuremath{\,\DD}}	% differential d

% A normal (total) derivative. If you supply an optional argument n,
% then you get the nth derivative: e.g., \deriv[2]{x}{t}
\newcommand{\deriv}[3][]{\ensuremath{%
	\ifthenelse{\equal{#1}{}}{\frac{\DD #2}{\DD #3}}
	{\frac{\DD^{#1} #2}{\DD #3^{#1}}}}}

% partial derivative
\newcommand{\pd}[3][]{\ensuremath{%
	\ifthenelse{\equal{#1}{}}{\frac{\partial #2}{\partial #3}}
	{\partial #2 / \partial #3}}}

% variational derivative
\newcommand{\varder}[2][]{\ensuremath{\deriv{}{t} \left( \pd{L}{\dot{#2}%
      \ifthenelse{\equal{}{#1}}{}{_{#1}}} \right)} - \pd{L}{#2%
      \ifthenelse{\equal{}{#1}}{}{_{#1}}} }

% variational derivative
\newcommand{\ele}[2][]{\ensuremath{\deriv{}{t} \left( \pd{\mathscr{L}}{\dot{#2}%
      \ifthenelse{\equal{}{#1}}{}{_{#1}}} \right)} = \pd{\mathscr{L}}{#2%
      \ifthenelse{\equal{}{#1}}{}{_{#1}}} }


\titlespacing\section{0pt}{12pt plus 4pt minus 2pt}{0pt plus 2pt minus 2pt}
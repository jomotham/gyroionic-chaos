\documentclass[12pt]{article}
% Math Packages
\usepackage{amssymb, amsmath, amsfonts, amsthm, mathtools, graphicx, tcolorbox, cancel, mathrsfs}

% Formatting Packages
\usepackage{cite, pdfpages, tikz, pgffor, float, pgfplots, import, xcolor, transparent, siunitx, caption, enumitem, titlesec, multicol, pifont, verbatim, fancyvrb, calc, xspace}


\usepackage{wrapfig,paralist,rotating,subfig}
\usepackage[textwidth=6in,textheight=8in]{geometry}

% Please use vector graphics in .pdf format where possible
% This is much preferable to large bitmap files
\DeclareGraphicsExtensions{{.pdf},{.jpg},{.png}}

\usepackage[english]{babel}
% \usepackage[utf8]{inputenc}
\usepackage[dvipsnames]{xcolor}
\usepackage[absolute]{textpos}

\usepackage{blindtext}


\usetikzlibrary{shapes.geometric}
\pgfplotsset{compat=1.18}
\pdfsuppresswarningpagegroup=1
\tikzset{every picture/.style={line width=0.75pt}}

\renewcommand{\baselinestretch}{1.5}

\newcommand{\eref}[2][]{%
	\ifthenelse{\equal{#1}{}}%
	{Eq.~(\ref{eq:#2})}%
	{Equation~(\ref{eq:#2})}\xspace}

% To refer to a figure, use \fref{<label>}, where <label> is the
% part that follows the 'fig:' in the label of the figure.
\newcommand{\fref}[2][]{%
  \ifthenelse{\equal{#1}{}}%
	{Fig.~\ref{fig:#2}}%
	{Figure~\ref{fig:#2}}}

% How to type vectors. If you don't like this style, you can just change
% the definition of \VEC!!
\newcommand{\VEC}[1]{\ensuremath{\boldsymbol{#1}}\xspace}

% unit vectors
\newcommand{\ux}{\ensuremath{\mathbf{\hat{x}}}\xspace}
\newcommand{\uy}{\ensuremath{\mathbf{\hat{x}}}\xspace}
\newcommand{\uz}{\ensuremath{\mathbf{\hat{x}}}\xspace}

% theta dot and double dot
\newcommand{\thd}{\ensuremath{\dot{\theta}}\xspace}
\newcommand{\thdd}{\ensuremath{\ddot{\theta}}\xspace}

% Derivatives. Some publications require the d in a differential
% to be set in roman. I've never liked that, but I have trained myself
% always to use these two macros, so if I need to switch, it is
% painless.
\newcommand{\DD}{\ensuremath{d}}		% differential d without leading space
\newcommand{\dd}{\ensuremath{\,\DD}}	% differential d

% A normal (total) derivative. If you supply an optional argument n,
% then you get the nth derivative: e.g., \deriv[2]{x}{t}
\newcommand{\deriv}[3][]{\ensuremath{%
	\ifthenelse{\equal{#1}{}}{\frac{\DD #2}{\DD #3}}
	{\frac{\DD^{#1} #2}{\DD #3^{#1}}}}}

% partial derivative
\newcommand{\pd}[3][]{\ensuremath{%
	\ifthenelse{\equal{#1}{}}{\frac{\partial #2}{\partial #3}}
	{\partial #2 / \partial #3}}}

% variational derivative
\newcommand{\varder}[2][]{\ensuremath{\deriv{}{t} \left( \pd{L}{\dot{#2}%
      \ifthenelse{\equal{}{#1}}{}{_{#1}}} \right)} - \pd{L}{#2%
      \ifthenelse{\equal{}{#1}}{}{_{#1}}} }

% variational derivative
\newcommand{\ele}[2][]{\ensuremath{\deriv{}{t} \left( \pd{\mathscr{L}}{\dot{#2}%
      \ifthenelse{\equal{}{#1}}{}{_{#1}}} \right)} = \pd{\mathscr{L}}{#2%
      \ifthenelse{\equal{}{#1}}{}{_{#1}}} }


\titlespacing\section{0pt}{12pt plus 4pt minus 2pt}{0pt plus 2pt minus 2pt}

\title{Phys 111 Computational Project}
\author{Aidan Gallade, Jonathan Holcombe}

\begin{document}
\maketitle

\section{Introduction}
Our chosen system consists of two charged particles, with charges $q_{1}$ and $q_{2}$  and masses $m_{1}$ and $m_{2}$ respectively, placed in a constant magnetic field, for simplicity's sake chosen to be $\mathbf{B} = B_{0}\, \hat{z}$.

In a system with just a single charged particle and a constant magnetic field along $\hat{z}$, the particle undergoes gyration in the $x$-$y$ plane about an axis parallel to $\hat{z}$ and approaches some constant speed in the $\hat{z}$ direction, as depicted in \fref{single-gyration}. If we take $|q|=|m| = 1$, then this terminal $z$-velocity is determined by $B_{0}$, and the radius of gyration is determined by both $B_{0}$ and the particle's initial velocity in the $x$-$y$ plane.

\begin{figure}[H]
    %\includegrahpics[width=0.8\linewidth]{filename.png}
    \caption{A single charged particle gyrating in the constant magnetic field $\mathbf{B} = B_{0}\, \hat{z}$.}
    \label{fig:single-gyration}
\end{figure}

By introducing a second particle, there will now be a Coulombic interaction between the two particles. Due to the Coulomb potential being proportional to $1/r$, we hope to see strong sensitivity to initial conditions and chaotic behavior as a result of introducing this second particle.

\section{Equations of Motion}
\subsection{Generalized Coordinates}
Since our system consists of two particles in three dimensions, this system has a maximum of six degrees of freedom. If we made certain assumptions about the mass or charge of the particle (such as the particles having equal charge-mass ratios), we might be able to reduce the number of degrees of freedom, but since we are not making any of these assumption, we are left with the initial six degrees of freedom, with generalized coordinates given by:
\[
(x_{1},y_{1},z_{1})\quad  (x_{2},y_{2},z_{2}).
\]

\subsection{Lagrangian(s)}
The Lagrangian of a single point charge $q$ in an electric and magnetic field is given by
\begin{align}
    \mathscr{L}(\mathbf{r},t) = \frac{1}{2}m \dot{\mathbf{r}}^{2} - q\, \phi(\mathbf{r},t) + q \,\dot{\mathbf{r}} \cdot \mathbf{A}(\mathbf{r},t) \label{eq:L_gen}
\end{align}
where $\phi(\mathbf{r},t)$ is the scalar potential of the electric field and $\mathbf{A}(\mathbf{r},t)$ is the vector potential of the magnetic field. Our system has a constant magnetic field of strength $B_{0}$ in the $\hat{z}$ direction, which has a vector potential of 
\begin{align}
    \mathbf{A}(\mathbf{r},t) = \mathbf{A} = -B_{0}y\;\hat{x}.
\end{align}
Since we have only two charged particles, the only electric field felt by one particle is the electric field of other particle, so the electric scalar potential of each particle is given by
\begin{align}
    \phi_{i}(\mathbf{r},t) = \frac{1}{4\pi\epsilon_{0}} \frac{q_{j}}{|\mathbf{r}_{i} - \mathbf{r}_{j}|} = \frac{1}{4\pi\epsilon_{0}} \frac{q_{j}}{d}
\end{align}
where we have defined $d = |\mathbf{r}_{i} - \mathbf{r}_{j}|$.

Substituting these potentials and our generalized coordinates into \eref{L_gen}, we get the Lagrangians of our two particles:
\begin{subequations}
    \label{eq:L}
    \begin{align}
    \mathscr{L}_{1}(x_{1},y_{1},z_{1}) = \frac{1}{2}m_{1} (\dot{x}_{1}^{2} + \dot{y}_{1}^{2} + \dot{z}_{1}^{2}) - \frac{1}{4\pi\epsilon_{0}} \frac{q_{1}q_{2}}{d} - q_{1}B_{0}y_{1}\dot{x}_{1}, \label{eq:L1} \\ 
    \mathscr{L}_{2}(x_{2},y_{2},z_{2}) = \frac{1}{2}m_{2} \dot(\dot{x}_{2}^{2} + \dot{y}_{2}^{2} + \dot{z}_{2}^{2}) - \frac{1}{4\pi\epsilon_{0}} \frac{q_{2}q_{1}}{d} - q_{2}B_{0}y_{2}\dot{x}_{2}, \label{eq:L2}
\end{align}
\end{subequations}
where $d$ is given by $\sqrt{(x_{2}-x_{1})^{2} +(y_{2}-y_{1})^{2} + (z_{2}-z_{1})^{2}  }$.

\subsection{Euler-Lagrange Equations}
Taking the Euler-Lagrange equations of these two Lagrangians, we get the following six differential equations:
\begin{subequations}
    \label{eq:ele1}
    \begin{align}
    \ele[1]{x}\implies&\ddot{x}_{1}=\frac{q_{1}B_{0}}{m_{1}}\dot{y}_{1}+\frac{q_{1}q_{2}}{4\pi \varepsilon_{0}m_{1}}\frac{x_{2}-x_{1}}{d^{3} }  \label{eq:elex1}   \\
    \ele[1]{y}\implies&\ddot{y}_{1}=-\frac{q_{1}B_{0}}{m_{1}}\dot{x}_{1} +\frac{q_{1}q_{2}}{4\pi \varepsilon_{0}m_{1}}\frac{y_{2}-y_{1}}{d^{3} }   \label{eq:eley1}       \\
    \ele[1]{z}\implies&\ddot{z}_{1} = \frac{q_{1}q_{2}}{4\pi \varepsilon_{0}m_{1}}\frac{z_{2}-z_{1}}{d^{3} } \label{eq:elez1} 
\end{align}
\end{subequations}
\begin{subequations}
    \label{eq:ele2}
    \begin{align}
    \ele[2]{x}\implies&\ddot{x}_{2}=\frac{q_{2}B_{0}}{m_{2}}\dot{y}_{2}-\frac{q_{1}q_{2}}{4\pi \varepsilon_{0}m_{2}}\frac{x_{2}-x_{1}}{d^{3} }   \label{eq:elex2}  \\
    \ele[2]{y}\implies&\ddot{y}_{2}=-\frac{q_{2}B_{0}}{m_{2}}\dot{x}_{2} -\frac{q_{1}q_{2}}{4\pi \varepsilon_{0}m_{2}}\frac{y_{2}-y_{1}}{d^{3} }    \label{eq:eley2}      \\
    \ele[2]{z}\implies&\ddot{z}_{2} = -\frac{q_{1}q_{2}}{4\pi \varepsilon_{0}m_{2}}\frac{z_{2}-z_{1}}{d^{3} }\label{eq:elez2}
\end{align}
\end{subequations}


\subsection{Non-dimensionalization}
To non-dimensionalize our equations of motion, we can combine the constants we have in our equations (e.g., $q,m,\varepsilon_{0},B_{0}$) to construct non-dimensionalized lengths and time:
\[
\left[ \left( \frac{B_{0}^2\varepsilon_{0}}{m} \right)^{1/3}   \right]  = \mathrm{m}^{-1} \qquad \left[ \frac{qB_{0}}{m} \right] = \mathrm{s}^{-1}. 
\]
Using these to write our generalized coordinates and their derivatives in terms of dimensionless variables, we get
\[
t = \tilde{t}\frac{m}{qB_{0}}, \qquad l = \tilde{l}\left(\frac{m}{B_{0}^2\varepsilon_{0}} \right)^{1/3}\hspace{-12pt},
\]
\[
\frac{dl }{dt }= \left( \frac{qB_{0}}{m} \right)\left( \frac{m}{B_{0}^2\varepsilon_{0}}   \right) ^{1/3}\frac{d^{2}\tilde{l} }{d\tilde{t}^{2} }, \qquad \frac{dl^{2} }{dt^{2} }= \left( \frac{qB_{0}}{m} \right)^{2}\left( \frac{m}{B_{0}^2\varepsilon_{0}}   \right) ^{1/3}\frac{d^{2}\tilde{l} }{d\tilde{t}^{2} }
\]
where the remaining derivatives all have the same non-dimensionalization constants. 

For simplicity, we pick $q_{1}$ and $m_{1}$ to replace the $q$ and $m$ in these constants (picking $q_{2}$ and $m_{2}$ would be equally valid, but we need to be consistent). Substituting these non-dimensional variables into our equations of motion in \eref{ele1} and \eref{ele2}, we arrive at 
\begin{subequations}
    \begin{align}
    &  \ddot{\tilde{x}}_{1} = \dot{\tilde{y}}  + \frac{1}{4\pi} \frac{q_{2} }{ q_{1}} \frac{\tilde{x}_{1}-\tilde{x}_{2}}{\tilde{d}^{3} }     \\
    &  \ddot{\tilde{y}}_{1} = -\dot{\tilde{x}}  + \frac{1}{4\pi}\frac{q_{2} }{q_{1}} \frac{\tilde{y}_{1}-\tilde{y}_{2}}{\tilde{d}^{3} }          \\
    &  \ddot{\tilde{z}}_{1} = \frac{1}{4\pi}\frac{q_{2} }{q_{1}} \frac{\tilde{z}_{1}-\tilde{z}_{2}}{\tilde{d}^{3} }   
\end{align}
\end{subequations}
\begin{subequations}
    \begin{align}
    &  \ddot{\tilde{x}}_{2}=\frac{q_{2}m_{1}}{q_{1}m_{2}}\dot{\tilde{y}}_{2}+ \frac{1}{4\pi}\frac{q_{2}m_{1}}{ q_{1}m_{2}}\frac{\tilde{x}_{1}-\tilde{x}_{2}}{\tilde{d}^{3} }    \\
    &  \ddot{\tilde{y}}_{2}=-\frac{q_{2}m_{1}}{q_{1}m_{2}}\dot{\tilde{x}}_{2}+ \frac{1}{4\pi}\frac{q_{2}m_{1}}{ q_{1}m_{2}}\frac{\tilde{y}_{1}-\tilde{y }_{2}}{\tilde{d}^{3} }          \\
    &  \ddot{\tilde{z}}_{2} = \frac{1}{4\pi}\frac{q_{2}m_{1}}{ q_{1}m_{2}}\frac{\tilde{z}_{1}-\tilde{z}_{2}}{\tilde{d}^{3} } 
\end{align}
\end{subequations}
These $q$ and $m$ factor on our equations for the second particle just specify its behavior differs first particle's in terms of their charge and mass ratios.


TODO: look into getting the ratio terms in both sets of DEs. Currently, with our non-dimensionalizing, we define the length and time scale based on particle 1 to have a length/time of "1", and define particle 2 relative to that. Instead, could we define the midpoint of their scales to be one, so that reciprocal coeffs get us the correct scales?

\section{Dynamics}

\section{Conclusion}


\end{document}